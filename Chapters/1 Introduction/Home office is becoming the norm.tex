


The popularity of home offices has risen again after fading in the 1990s.
Americans' work habits and environments have changed dramatically since Coronavirus
closed offices nationwide, resulting in millions of professionals suddenly working from home.
Many are looking at the likelihood of long-term teleworking
to remain an integral part of the work environment long after the pandemic.\cite{BibEntry2020Jun}

Unsurprisingly, a survey report conducted by owl labs suggests that workers who were working from home
reported to be happy 22 percent more than workers who work in an on-site office environment.\cite{BibEntry2022Nov}

The pandemic is not the only reason teleworking is becoming common:
With the help of modern technological work tools,  more and more people are able to work from no matter where.
This opens up opportunities for people who may not have access to traditional office spaces,
and it also allows for more flexible working arrangements.


One of the benefits of working from home is that it blurs the line
between work and private life.
This can be a good thing, as it can make it easier to transition between work and relaxation.
On the other hand, working from home can bare some downsides too \cite{Marsh2022Mar}.

According to a study conducted by Joblist \cite{Joblist2022Nov},
working remotely led to increased distractability in 53.1 percent of respondents, claiming
that they found it to be difficult to stay focused on the task at hand.

%social media is bad for us

Research has shown that digital distraction has become the primary factor for procrastination\cite{Lu2014Dec}
and Social Media is likely to play a big part of that as there
is no doubt that social media can be bad for our productivity.\cite{Vithayathil2018Jan}
In recent years, social media has become an increasingly prevalent part of our lives.
According to a study by the
Pew Research Center, as of April 2021,  71 percent of
American adults use social media platforms such as
Facebook, Instagram, LinkedIn, and Twitter.\cite{PewResearch2022Nov}

A study conducted by the University of California,
Irvine found that social media is the main distraction at work today.
The study found that employees were interrupted by social media
notifications an average of every 10.5 minutes
, and it took an average of 25 minutes to return to their original task.\cite{Mark2008Jan}


%why productivity is important

David McClelland argued that productivity is important to life satisfaction
because it is a key ingredient in achieving goals and meeting needs.
He stated that people who are productive are happier and more satisfied with their lives
because they feel like they are accomplishing something.\cite[p.~159]{McClelland1961}
Furthermore, John W. Kendrick argues,
that personal productivity is not only a private matter, but a societal one.
John Kendrick's argument is that productivity is not only a personal matter, but that it is also a societal one.
This means that if individuals are successful and productive, then the whole society will be as well.
This is because a productive society is more likely to be prosperous and thrive than one that is not.
\cite{Kendrick1987May}


% It is important to rethink productivity
Under modern working conditions, it is becoming increasingly difficult to separate work and personal life,
as technology allows for more flexibility in where and when we can complete tasks.



Respondents working from home also reported that they would check their work devices
more frequently than they did before going remote
and they regularly work past normal office hours. In fact,
70.9 percent of managers self-reportedly worked past normal office hours
on a regular basis since working from home

Consequently, this can lead to an increase in stress levels and fatigue
amongst employees as many are struggling to juggle work,
home and family obligations.


Craig Brood characterized this phenomenon as technostress.
The idea of technostress is that the computer revolution has had
a number of negative consequences for human health,
including increased levels of stress and anxiety.
Technostress can cause us to become stressed out and overwhelmed,
and it can also lead to other problems like addiction and distraction.
We need to be aware of the dangers of technostress and take steps to avoid it,
especially in an environment where work and private happen simultaneously in a digital world.

By being productive, we can
create a sense of satisfaction and fulfillment in our lives \cite{Csikzentmihalyi1990}
but it is difficult to be productive in an environment
that isn't conductive to it.


"To overcome the anxieties and depressions of contemporary life, indi-
viduals must become independent of the social environment to the degree
that they no longer respond exclusively
in terms of its rewards and punishments.
To achieve such autonomy, a person has to learn to provide rewards
to herself. She has to develop the ability to find enjoyment and purpose
regardless of external circumstances."\cite[p.16]{Csikzentmihalyi1990}

We argue that it is more important now than ever to design systems for a productive life, especially
because at the same time, the new work and study environments that we find ourselves in as a consequence of the
digital revolution, require us to become more self-structured. This absence of an imposed guidance means that the competent worker has to create the order himself and learn to self-regulate \cite{Piers2007}
regardless of the physical location and circumstances and create
products that inspire us to be productive. We need to form
environments that are conducive to productivity,
and we need to have attitudes about productivity that support us in our endeavors.
We also need to have social media platforms that don't distract us from our goals,
and we need to be aware of the dangers of technostress so that we can take steps to avoid it.
