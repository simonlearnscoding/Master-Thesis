%TAKE A LOOK AT THIS

Digital Nudges \cite{Leonard2008Dec} have been described
as something aiming to “alter people’s behaviour predictably
without forbidding any option”.
Some Have also called this libertarian paternalism \cite{Leonard2008Dec}

% How is that even relevant to anything?
"Humans are not exactly lemmings, but they are easily influenced by the statements and deeds of others." \cite{Leonard2008Dec}



The term nudge has been popularized by behavioral scientist (author name) and (name).

( name)defines a nudge as "any aspect of choice architecture that alters peoples behavior in a
predictable way without forbidding any options or  significantly changing their economic incentives.
To count as a nudge, the intervention must be easy and cheep to avoid. Putting the fruit at eye level counts as a nudge, banning junk food does not"

To take a closer look at this definition it is important to first examine what exactly is meant by the term "choice architecture"

\subsection{Choice architecture}
In a very broad sense choice architecture is the practice of influencing an individuals behavior by changing the context in which they make a decision. In a supermarket for example it is possible to influence a buyers behavior by changing the order of the shopping items. Customers who pass through a section where apples are placed at eye level are proven to be much more likely to buy that product for example, then those who pass through another section where that is not the case. Choice architecture as a concept in itself does not imply any intentional effort to influence someones behavior, as intentional choice architecture can take place too, for example when an actor places the items without any intention to influence choices yet still influence choices because of the unintentional placement.
However, a  nudge on the other hand, does imply an intentional action to influence the users behavior. Key difference between the two therefore is whether an attempt to change behavior in a predictable way, without forbidding any options or significantly changing economic incentives. "to count as a mere nudge, the intervention must be easy and cheep to avoid. Nudges are not mandates"

To  summarize, these are the three main criteria that must be fulfilled in order for an Intervention to count as a nudge:
* cannot forbid any options
* cannot change economic incentives
* must be easy and cheap to avoid

hello
\section{The ethics of nudge and gamification}

\subsection{Nudge and freedom of Choice}
The be fundamental beliefs make up the foundation of the notion of libertarian paternalism.
the idea is that people always go the path of least resistance when it comes to making  decisions.

\subsection{libertarian paternalism}
According to the  philosophy of libertarian paternalism, like it's name suggests, it is possible and morally plausible to maintain a form of paternalism that maintains the individuals freedom of choice. It insists that it is important to maintain the individuals freedom of choice first and forePHEFT but at the same time it acknowledges that human thinking can be prone to  various biases and we might therefore not always be in the position to make choices that are into our own interest.


According to libertarian paternalism it is immoral to  forbid the wrong option entirely because this belief presupposes that it is not good to



Nudge:

https://www.youtube.com/watch?v=33ODewJG8Fg
https://www.edelweissmf.com/Files/Insigths/booksummary/pdf/EdelweissMF_BookSummary_Nudge.pdf

Nudging Students:
https://learninganalytics.upenn.edu/ryanbaker/IV2021_Nudges.pdf


Pervasive Technology:
https://dl.acm.org/doi/abs/10.1145/764008.763957
Eded