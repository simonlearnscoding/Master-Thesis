\subsection{The Matching Law}
%TODO matching law illustration

The Matching Law, also known as the Principle of Reinforcement or Herrnstein's Law of Effect,
is an important concept in behavioral economics.
It states that the strength of a particular response (e.g. pulling a lever)
is directly proportional to the strength of the stimulus (e.g. being given food).
It states that an individual's preference for one alternative over another can be accurately predicted
if the frequency (or probability) and magnitude of rewards (or reinforcers) associated with each
alternative are known.
Simply put, it states that when two or more alternatives are present, the individual will tend to select
the alternative with the higher ratio of reward to cost.
This concept is especially important in situations where an individual must decide between competing activities, such as work and leisure. 
It can also be used to understand how individuals make choices in markets with multiple products, services, or stores. 
The Matching Law provides an important insight into how people make decisions, and can be used to better inform public policy decisions. As such, it is a valuable tool for economists and policymakers alike.
This law is also applicable to procrastination,
as it suggests that the more motivation and value an individual places on a task,
the more likely they are to put forth effort towards achieving their goals.\cite{Poling2011Apr}

This concept has been connected to the concept of System One and System Two Thinking,
introduced by Nobel Prize winning economist Daniel Kahneman.







