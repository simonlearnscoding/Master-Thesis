\subsection{Classical conditioning}

Classical conditioning is a learning style that became famous in the late 19th century
Ivan Pavlov's experiments with dogs have since been applied to many areas of
psychology.

Pavlov fed the dogs and measured their saliva
answer. Pavlov started ringing the bell just before offering the food.
After the dog has been given a  number of food-bell presentations,
Pavlov noticed that the dog would salivate when the bell rang, even
though no food was offered. Pavlov called this the conditioned reflex.
The dog had learned to associate the bell with food, and the bell
became a conditioned stimulus.

Classical conditioning occurs when instinctive drive responses are associated
with new stimuli.
A stimulus to which an organism responds without training is called
primary or unconditioned stimulus. A stimulus that does not elicit a
response until it is associated with a primary stimulus is called
secondary or conditioned stimulus. The response to a primary stimulus
is called unconditioned response. The response to a secondary stimulus
is called conditioned response.
By pairing an unconditioned stimulus with a previously neutral stimulus, it is possible to create new connections
between stimuli and responses that may not have existed before. Through careful manipulation of the environment,
classical conditioning can be used to alter behavior in a variety of ways.
By understanding the principles of classical conditioning, it is possible to develop more effective therapeutic techniques for individuals dealing with procrastination.\cite{Rehman2022Aug}
Classical conditioning is an example of System One (or "automatic") thinking.
This type of learning occurs quickly and without conscious effort,
allowing a subject to respond to stimuli in the environment without having to think about it.
By contrast, System Two (or "controlled") thinking involves more deliberate mental processing, as well as the application of specific strategies to solve problems. Whereas System One learning is an unconscious process, System Two thinking requires active engagement and effort. Classical conditioning involves pairing a previously neutral stimulus with an unconditioned stimulus that produces a particular kind of response. This happens quickly, without the need for conscious thought, which makes it an example of System One thinking. By associating specific stimuli with particular responses, classical conditioning can be used to influence behavior in a variety of ways.  It is important to note, however, that the effects of classical conditioning are short-lived and may not persist over time. Therefore, it is important to use this method in combination with other strategies such as System Two thinking in order to achieve more lasting

