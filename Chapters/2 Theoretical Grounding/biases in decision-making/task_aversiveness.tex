

Task aversiveness is the degree to which an individual perceives a task to be difficult, unpleasant, or challenging. It is typically assessed by measures of cognitive difficulty, effort expenditure (physical or mental), and frustration. Task aversiveness may also encompass subjective evaluations of task value, relevance, and importance. Aversiveness can have an impact on task performance, with highly aversive tasks being more prone to errors and decreased productivity. Furthermore, prolonged exposure to aversive tasks can lead to burnout, with individuals exhibiting diminished motivation and engagement. Factors such as task complexity, ambiguity of task requirements and expectations, regimen of reinforcement or reward strategies employed for the individual completing the task may all influence the degree of task aversiveness experienced by an individual.

The term “task aversiveness” was first introduced by psychologist Julian B. Rotter in 1954. He defined the concept as “the degree to which an individual perceives a task to be difficult, unpleasant, or challenging”. Since then, the concept has been studied extensively and expanded upon.

Rotter's original framework considers both cognitive and affective factors in assessing task aversiveness. Cognitive factors include individual judgment of the difficulty, complexity, or effort of completing a task, while affective factors such as frustration, boredom, anxiety are also taken into consideration.
\cite{Rotter1954}