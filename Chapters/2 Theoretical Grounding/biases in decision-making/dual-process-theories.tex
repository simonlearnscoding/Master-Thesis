
Dual-process theories vary greatly but generally share the overarching structure of positing two types of human information processing — automatic and nonautomatic — in explaining and predicting human behavior.

Automatic processing is largely unconscious and referred to as type 1, while nonautomatic processing is conscious and effortful, and referred to as type 2. Type 1 processes are said to be involuntary and automatic, triggered by an environmental stimulus or cue in an almost reflexive manner. They involve little cognitive load and require minimal attentional resources. In contrast, type 2 processes are said to be voluntary and controlled, requiring conscious effort and attentional resources. Type 1 is thought to be relatively inflexible and type 2 more flexible. The dual-process model thus suggests that behavior results from a combination of both types of processing — with either type leading or contributing depending on the situation. It also indicates that different types of behavior — e.g., a cognitive task versus an emotional response — may be associated with different types of processing. For example, type 1 processes may be dominant when an individual is responding to an emotion-eliciting situation, while type 2 processes may be activated more during tasks requiring analytical reasoning or decision making.\cite{Hansen2022Dec} 


In the context of this research paper, the thinking fast and slow framework, which was developed by the renowned cognitive psychologist and Nobel Laureate in Economics, Daniel Kahneman, as well as the thinker and doer model proposed by Thaler, will be explored.

