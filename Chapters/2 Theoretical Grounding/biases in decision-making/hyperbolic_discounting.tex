The concept of  hyperbolic discounting was first described by Ainslie, it is
implied in Richard Herrnstein's "Law of Matching".
Hyperbolic discounting is when people choose a smaller, sooner reward over a bigger, later reward. This happens because the person values the smaller, sooner reward more than the bigger, later reward. Hyperbolic discounting can lead to procrastination because it leads us to prefer taking the easier, quicker route instead of the route that will be more beneficial in the long run.
Ainslie proposed that when given a choice between two rewards, people tend to choose the sooner, smaller reward over the delayed, larger one, because humans (and animals) irrationally place more value on the current reward rather than the potential one in the future.
This phenomenon can be explained using two factors: temporal distance and amount of reward.
He observed this phenomena by making Pigeons choose between 2 and 4 seconds of grain approach in a separate two-button test procedure, with the larger dose always being presented 4 seconds later than the smaller one. All subjects reversed their preference from small early reinforcement to large late reinforcement, as the delay between selection and availability of small reinforcement varied from 0.01 s to 12 s. The priority-reversed delay values ​​were nearly consistent with the adaptive law adapted to the delay gain.
Hyperbolic discounting can lead to procrastination as it encourages people to take the easier, quicker route rather than the one that will be more beneficial for them in the long run.
\cite{Ainslie1981Dec}



