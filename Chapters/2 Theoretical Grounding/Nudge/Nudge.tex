%TAKE A LOOK AT THIS
Richard H. Thaler and Cass R. Sunstein focus on how small changes in the environment can influence people's decisions and behavior in their book "Nudge. The authors explore the concept of "choice architecture", or the idea that our decisions are influenced by the choices we are presented with. Nudging is a form of behavioral economics that uses subtle environmental cues to influence people's decisions. Richard Thaler himself, defines a nudge as "a feature of choice architecture that influences people's decisions in a specific, predetermined direction without limiting any options or significantly modifying their economic incentives." \cite[p.14]{Thaler2008}

They argue that certain designs, structures, and contexts can be used to "nudge" people towards making certain decisions that are ultimately in their best interests. For example, displaying healthy food options at eye level or making them more accessible in store aisles can be considered a nudge because it makes it easier for people to choose the healthier option. Similarly, providing information about energy-saving initiatives may prompt people to take part in those activities. Nudging has been found to be an effective way of promoting healthier behaviors without limiting choices or significantly altering economic incentives. The effectiveness of nudging lies in its ability to change behavior without resorting to more drastic measures such as legal or financial incentives. Nudges have been found to be effective in achieving a variety of goals, from reducing energy consumption and encouraging healthier dietary choices to improving compliance with safety regulations. \cite{Thaler2008}

