%THE POWER OF DEFAULTS

Thaler explains that by default humans choose the option that requires the least amount of effort, or in other words, the path of the least resistance. He argues that if there is a default option to choose in an interaction, the user will most likely go for it, especially, if it is in somewhat implied, that this default option is the "normal", or the recommended option to choose. He further claims that most successful organizations have already learnt on how to capitalize on what he refers to as the "power of defaults" by making customers stay subscribed to their service, for example.

 Two motives usually stand behind the usage of the power of default: the helpful and the self-serving.
 An example of a helpful usage of this principle is the option of a default installation on a Software.

 An example for the self-serving category is when the company sends their customers emails by default, informing them about new upcoming products.

The alternative to the default choice strategy is called the required choice.
An example for a required choice design is a form with a multiple choice selection where all choices are unselected and the user is forced to make a choice on his own.
A required choice architecture can be useful if the choice architect wants to avoid to create a morally questionable choice design, but it's downside can be that sometimes the user doesn't want to be confronted with complicated choice
 \~\cite[ch. 1.5, p.3-6]{Thaler2008}
