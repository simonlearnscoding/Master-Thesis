"If you indirectly influence the choices other people make, you are a choice architect." \~\cite[ch. 1.5, p. 3]{Thaler2008}
% How is that even relevant to anything?

\subsection{Choice architecture}
In a very broad sense choice architecture is the practice of influencing an individuals behavior by changing the context in which they make a decision. In a supermarket for example it is possible to influence a buyers behavior by changing the order of the shopping items. Customers who pass through a section where apples are placed at eye level are proven to be much more likely to buy that product for example, then those who pass through another section where that is not the case. Choice architecture as a concept in itself does not imply any intentional effort to influence someones behavior, as intentional choice architecture can take place too, for example when an actor places the items without any intention to influence choices yet still influence choices because of the unintentional placement.
However, a  nudge on the other hand, does imply an intentional action to influence the users behavior. Key difference between the two therefore is whether an attempt to change behavior in a predictable way, without forbidding any options or significantly changing economic incentives. "to count as a mere nudge, the intervention must be easy and cheep to avoid. Nudges are not mandates"

To  summarize, these are the three main criteria that must be fulfilled in order for an Intervention to count as a nudge:
* cannot forbid any options
* cannot change economic incentives
* must be easy and cheap to avoid
