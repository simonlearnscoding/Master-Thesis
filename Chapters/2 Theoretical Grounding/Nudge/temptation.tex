
% Temptation
Thaler describes temptation as a "hot and cold state". In the hot state, our intuition and emotions are
more likely to drive our decision-making processes. This can lead people to make decisions based on impulse rather than through thoughtful analysis.
The hot state is often associated with a sense of urgency or impulsivity.
On the other hand, in the cold state our decision-making processes are guided by rational thinking and logical reasoning.
In this state, people are more likely to make decisions which reflect their
long-term goals rather than short-term desires. The cold state is often associated with a sense of calmness and patience.
Thaler suggests that the cold state is the optimal state for making decisions.
However, the hot state is often more appealing to people, as it is associated with a sense of urgency and excitement.
This is why people often find it difficult to resist temptation.\cite[ch 1.3, p.4]{Thaler2008}

% Doer and thinker
Thaler uses the analogy of two semiautonomous selves, a far-sighted "Planner" and a myopic "Doer",
to illustrate how we can better understand issues of self-control.
The Planner is the part of us that takes a longer view,
assessing our decisions in terms of the long-term goals we have set for ourselves.
Our 'planner' guides our intuition and decision-making processes through our
emotions, experiences and heuristics.
The Doer, on the other hand, is focussed more on immediate gratification and pleasure;
It is fast, automatic, effortless and instinctive – the ‘gut feeling’.
it makes decisions which are more likely to be influenced by temptation and short-term reward.
\cite[ch 1.3, p.4]{Thaler2008}

Thalers Framework of the Doer and Planner is a useful tool for understanding the relationship between our intuition and our rationality.
It is also a useful tool for understanding the relationship between our emotions and our rationality.
The Doer and Planner are not two separate entities, but rather two different aspects of the same entity.
They are both part of our intuition, but they are not the same.
The Doer and Planner are very similar, if not the same concept of System 1 and System 2 thinking, as described by
Daniel Kahneman.\cite{Kahneman2011}




% example of temptation
Thaler once used this technique to assist his young coworker David, who was having trouble writing his thesis.
Although David's inner Planner was aware that he needed to finish his thesis, his inner Doer was always putting off the tedious task of writing it up since he was working on other, more interesting tasks.

Thaler made the following bargain with David.
David would issue Thaler a succession of $100 checks, each due on the first of the following few months.
If David did not place a copy of a fresh chapter of the thesis under his door by the end of the subsequent month, Thaler would cash each cheque.
Four months later, David finished his thesis on time and never missed a deadline.\cite{Thaler2008}
