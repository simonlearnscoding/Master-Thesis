%Designing for reducing procrastination on side projects https://dl.acm.org/doi/10.1145/2686612.2686673

%this paragraph needs to be rewritten


% TODO: gamification in elearning systems: https://link.springer.com/chapter/10.1007/978-3-319-20609-7_56
%TODO: gamification for smarter learning: https://link.springer.com/article/10.1186/s40561-015-0017-8
"""
We see a growing body of literature on productivity tracking technologies designed to help people
track their productivity to improve self-awareness,
which could lead to enhanced productivity [21, 27, 37, 38, 44, 52].
Because information devices (e.g., computers, smartphones)
have become an integral part of people's work, as well as a source of distraction,
many productivity tracking systems enable people to record device usage behaviors to provide insights
into their usage patterns. Some systems consider screen time as distractive, thereby restricting specific apps
(e.g., [26]) or locking smartphones for a specified duration
(e.g., [25, 29, 30]) when people need to focus on their tasks.
%mention the papers you read

These systems predominantly incorporate automated tracking, and thus rely on the relatively simple measurements that can be captured. For example, many of these tools capture the usage duration of each application (e.g., [21, 27, 32, 43, 52]) or the device (e.g., [44]), and some of them calculate a productivity score derived from the ratio of productive application usage to total computer usage  [27, 43]. Commercial tools for developers’ productivity usually track a developer's programming activities such as interactions with the integrated development environment (e.g., [11, 51]).

Although automated tracking reduces the capture burden and collects behavioral data with high granularity,
this approach has three main limitations.
First, automated tracking does not capture productive activities that people do without devices (e.g., ad-hoc meeting).

Second, automated tracking does not always correctly infer a person's intention of using applications, websites, or devices.
For example, people can use a video chat application (e.g., Skype, Hangout) or visit an online shopping site (e.g., Amazon, Ebay)
both for work and for personal purposes. Most importantly, the duration of the app use is not reflective of a person's perceived productivity. For example, working on a Word document for a long time may not be an indication of productive writing. In this light, we need to understand how people assess their own productivity and incorporate them in the design of productivity tracking systems. We thus set out to collect self-reported data on activities that are perceived to be productive, along with contexts and their reasoning through a diary study."""
\cite{Kim2019May}




% PRODUCTIVITY TOOLS ( This will Be attached to Tools on the market )


Productivity tracking tools often base productivity decisions on  time spent interacting with work-related applications. \cite{Kenton2022Aug}

Given that Smartphones and other digital devices can be a major distraction in the workplace.\cite{Mark2008Jan} Some systems consider screen time as distractive, thereby restricting specific apps or locking smartphones for a specified duration when people need to focus on their tasks. This is often done to minimize the amount of time employees spend on social media or playing games instead of working. \cite{Kim2019May}


People who use productivity tracking systems are able to record their device usage behaviors in order to gain insights into their usage patterns. By understanding how they use their devices, people can be better equipped to regulate them and reduce the impact of distraction on their work. Leading scientists in the field of procrastination, such as Piers Steel, John Perry and Timothy Pychyl, have studied how device usage has impacted procrastination levels. As a result, they suggest that by tracking device usage over time through productivity tracking systems, individuals can better manage the distractions created by their devices and increase their overall productivity.

% TOOLS PRODUCTIVITY NUDGE 

Some of the most common productivity tools for nudge include software such as RescueTime, procrastination.com, and focus@will. These tools can be used to help you create a custom workflow that works for you, and can be very effective in overcoming procrastination.
The state of the art of HCI and productivity is constantly evolving, and there are a variety of tools and resources that can help you stay productive and focused. Some of the best tools for this purpose include software such as RescueTime, procrastination.com, and focus@will. These tools can be used to help you create a custom workflow that works for you, and can be very effective in overcoming procrastination.
Research on nudge and productivity suggests that these tools can be effective in overcoming procrastination.
