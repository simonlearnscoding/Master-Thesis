%DIFFERENCE BETWEEN PROFESSIONAL AND PERSONAL PRODUCTIVITY

Productivity is a very broad chapter and while there is not
one definition of productivity that general consensus agrees upon,
the most common definition of productivity is the amount of
output per input or,very generally speaking, Productivity can be defined as the amount of work
accomplished in a specific time period,
or, by how many tasks are completed in a day.

"At the corporate level, productivity is
a measure of the efficiency of a company's production
process, it is calculated by measuring the number of
units produced relative to employee labor hours or by
measuring a company's net sales relative to employee labor hours."
\cite{Kenton2022Aug}

In order to deconstruct the nebulous nature of the concept of productivity, Kim et al 

\cite{Kim2019May} examined how knowledge workers conceptalize personal productivity in both work, as well as non-work contexts.
The researchers conducted a two-week diary survey and then semi-structured interviews with 24 knowledge workers.
The participants in the study recorded what productive activities they had engaged in and why they believed those activities were productive. They did this in order to provide evidence for the usefulness of productivity ratings.

Surprisingly, they reported a wide range of productive activities beyond a typical desk job - from "hanging out with dad" to "going to the hairdresser".


% FOUR FACTORS THAT CONSTRUCT PERSONAL PRODUCTIVITY

Kim et al \cite{Kim2019May} identified four factors constituting the work productivity:

Concrete output and progress, and conceptual performance refer to material and immaterial forms of task performance, respectively.

By producing tangible results, such as implementing  new features, creating design artifacts, or documenting  design ideas, participants began to perceive the task as productive. Solving problems or making tangible progress from meetings was also considered productive (for example, making decisions on agenda items). Additionally, conceptual outcomes such as gaining insight, developing ideas, and acquiring knowledge contributed to productivity. In his exit interview, P9 said, "When I learn a new concept, I am proud of it, but that sense of  progress may be illusory.. Many such cases have turned out to be very productive. ”


Both quality and quantity were important factors in evaluating the performance of an individual's work. Poor quality output  made participants feel less productive and stressed. This is because it can impede progress or require additional tasks. P19s were sensitive to the timeliness of the counseling session (e.g., 'Insights [given to the client and completed on time] - were very productive'). Personal condition during or after the activity influenced how participants perceived their performance and productivity. Participant he conquered three major states.
Attention and distraction, emotional  and physical states. Attention refers to the state in which participants were able to focus on their task. Participants recorded various psychological responses to the task. B. Tired, happy, or depressed.

Participants appreciated activities that could benefit their careers, relationships, well-being, or finances. For example, going to the gym  when tired was thought to be productive, regardless of exercise intensity.Happiness and financial reward benefits were associated with productivity. Participants considered their activities to be productive if they had a positive impact on their relationships with family and colleagues. Financial activities such as refinancing and early ticket purchase were perceived as productive as they provided economic benefits to participants. The productivity of religious activities was measured by the spiritual benefits they
Personal productivity takes into account the personal importance, the personal value that the individual puts on the tasks that are
being completed.
To  conclude, personal productivity is not just a measure of economic performance but how efficiently one is able to complete tasks that are subjectively important to oneself and therefore getting him or her closer to living the life he or she wants. We argue therefore, that personal productivity is one of the most important aspects of life satisfaction.

% PRODUCTIVITY MEANS FREEDOM
Personal productivity is a matter of personal freedom, given that the measurement for success is ones own prioritization, which allows one to be in control of his or her life and achieve what is important to him or herself.
An individual is free to choose the tasks that he or she wants to complete and the way that he or she wants to complete them.  One is also free to set his or her own goals and to determine his or her own priorities. This allows the individual to be in control of his or her own life and to achieve what is important to him or herself.

% PRODUCTIVITY IS IMPORTANT
Personal productivity therefore has an intrinsic value, not just to achieve goals for the sake of achieving them but to improve mental health, reduce stress and improve sleep quality.
Productivity is an important skill for everyone to have, whether you are at work or at home. By understanding the differences between work and personal productivity, you can better focus on achieving your goals in a more effective way
%



