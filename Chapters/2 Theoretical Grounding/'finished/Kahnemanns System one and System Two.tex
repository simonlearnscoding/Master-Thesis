Daniel Kahneman explores the concept of procrastination in his book Thinking, Fast and Slow and offers
insight into why we so often put off tasks.
Kahnemann describes two distinct systems that govern our decision-making: System One and System Two.

System one and system two thinking refer to the two distinct ways in which humans process information.
System one is often referred to as “fast” or “intuitive” thinking, while system two is known as
“slow” or “computational” thinking.

System one thinking involves quick reactions to stimuli and does not require a lot of cognitive effort.
It is based on the neural pathways that have been built up over time, such as Pavlovian conditioning.
System one thinking is used for tasks that are familiar or have been learned through repeated experience,
such as recognizing faces or predicting what someone will say next in conversation.


System two thinking, on the other hand, requires more cognitive effort
and is used for tasks that are new or unfamiliar.
It takes more time to process information through system two thinking
as it involves more complex calculations and
problem-solving.
It is often used for activities such as
solving mathematical equations or writing a research paper.


System one and system two thinking both have their advantages and disadvantages.
System one thinking can be useful for quickly responding to a situation or making decisions without much thought,
but it can also lead to biases and errors in judgement. System two thinking requires more cognitive effort but
has the potential to yield more accurate results.

Generally, our instinctive System 1 is the one that makes us procrastinate,
causing us to prefer short-term rewards, such as pleasure and immediate gratification over long-term goals.
At the same time, System 2 can help us put in place strategies that combat this tendency towards procrastination
by forcing ourselves to focus on the task at hand.\cite[p. 1 - 13]{Kahneman2011}

Piers Steel argues that procrastination occurs when system one becomes too dominant,
and system two is not able to override system one's decision.\cite[chapt. 3, p. 3]{Steel2007}