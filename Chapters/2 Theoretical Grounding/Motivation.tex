
\todo definition of motivation


To be motivated means to feel an impulse to do something, but research suggests that motivation is not such a unitary phenomenon. There are not only different amounts but also different kinds of motivation. \cite{Ryan2000Jan}
Self-determination theory is the theory that describes motivation as two distinct types of motivation: Autonomous (which is being regulated through natural and internal processes such as inherent satisfaction) and Controlled motivation (regulated through external circumstances and demands) \cite{Lawman2013} these distinctions are more commonly known as intrinsic and extrinsic motivation \cite{Legault2016Nov}

\subsubsection{Intrinsic Motivation}
Psychology has a wealth of literature explaining why intrinsic motivation is essential to cognitive growth and organization.

Intrinsic motivation is a fundamental concept in developmental psychology.

Contemporary psychology makes a distinction between two main types of motivation, extrinsic and intrinsic motivation.

\subsubsection{Extrinsic Motivation}
"Extrinsic motivation is a construct that pertains whenever an activity is done in order to attain some separable outcome. Extrinsic motivation thus contrasts with intrinsic motivation, which refers to doing an activity simply for the enjoyment of the activity itself, rather than its instrumental value. "
\cite{Ryan2000Jan}


Social influence theory is a theory in psychology that talks about how people are more likely to do whatever they see as being the norm. It states that people have a tendency to change their behavior according to those around them, and those nearby have stronger effects than those further away.


#intrinsic/extrinsic motivation https://www.sciencedirect.com/science/article/pii/S0361476X99910202
# Self Determination Theory
https://link.springer.com/referenceworkentry/10.1007/978-1-4419-1005-9_1620