%What are the commonalities between Nudge and Gamification??

The word "gamification" was coined by Nick Pelling in 2002,
but it only became popular in 2010. Gamification refers to the use
of game elements and design techniques in non-game contexts.
This can include adding points, levels, and rewards to tasks as a
way to motivate people. \cite{burke2011gamification}

Blohm described Gamification as a
“persuasive technology that attempts to influence user
behavior by activating individual motives via game-design
elements” \cite{Blohm}.

In this sense, Gamification seems to show a lot of parallels
with the concept of nudging.



Nudging and Gamification have proven to provide a great increase in motivation if applied correctly. For example, there has been a study, where two groups of computer science students had to conduct various tasks on git,  67 percent of the members of the group, where gamification aspects  were implemented, reported an increased  motivaton to fulfill their tasks. It  is remarkable that it seems like there was no impact on age or gender on the percieved effects of this  phenomenon.
although it must be said that this study  marked one of the potential flaws of gameification by design comonly known as "gaming the system". An  observed phenomenon where the Metrics of success in a gamified system  become the main goal for the "players", a topic which shall be further examined later on in this thesis. \cite{Ozdamli2021Aug}





%TODO citing needed

There are a few key differences between nudge and gamification. Nudge theory is based on the idea that people can be subtly influenced into making certain choices, while gamification relies on more explicit rewards and punishments to motivate people. Nudge theory is also more focused on changing behaviour, while gamification is more focused on increasing engagement or motivation.




The paper discusses the difference between nudge and gamification, and how each can be used to increase productivity or engagement. It cites research from John Perry, Piers Steel, and Tim Pychyl on procrastination, and explains how each researcher has developed different strategies for overcoming it. Finally, it summarizes the motivation equation developed by Piers Steel, which suggests that there are four factors that influence motivation: ability, motivation, opportunity, and expectancy.


