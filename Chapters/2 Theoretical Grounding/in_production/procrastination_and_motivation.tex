
% THE RELATIONSHIP BETWEEN MOTIVATION AND PROCRASTINATION
There is a lot of research on the relationship between procrastination and motivation,
and it is still an ongoing area of exploration. Some of the most recent research on this topic has been conducted by Piers Steel, who has developed a model known as the "motivation equation". This model suggests that there are four factors that influence motivation:


- Ability: The ability to complete a task

- Motivation: The desire to complete a task

- Opportunity: The opportunity to complete a task

- Expectancy: The belief that completing a task will lead to a desired outcome


According to this model, motivation is influenced by both ability and expectancy. If you believe that you are not able to complete a task, or if you do not believe that completing a task will lead to the desired outcome, your motivation will be reduced.




How do procrastination relate to motivation?

Both procrastination and motivation are related to goal setting. Procrastination occurs when people do not set goals for themselves, or when they set unrealistic goals that are difficult to achieve. Motivation occurs when people set realistic goals and have the ability to achieve them.


According to Tom Perry, the relationship between motivation and procrastination is complex. Motivation can lead to procrastination if people set unrealistic goals for themselves, or if they do not believe that they are able to complete a task. However, procrastination can also lead to motivation if people are able to overcome their fear of failure and start taking small steps towards their goal.


Procrastination is generally seen as a common behavioral tendency, and there are a growing number of literatures discussing this complex phenomenon.


