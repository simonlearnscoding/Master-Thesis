\subsection{Classical conditioning}

Classical conditioning is a learning style that became famous in the late 19th century
Ivan Pavlov's experiments with dogs have since been applied to many areas of
psychology.

Pavlov fed the dogs and measured their saliva
answer. Pavlov started ringing the bell just before offering the food.
After the dog has been given a  number of food-bell presentations,
Pavlov noticed that the dog would salivate when the bell rang, even
though no food was offered. Pavlov called this the conditioned reflex.
The dog had learned to associate the bell with food, and the bell
became a conditioned stimulus.

Classical conditioning occurs when instinctive drive responses are associated
with new stimuli.
A stimulus to which an organism responds without training is called
primary or unconditioned stimulus. A stimulus that does not elicit a
response until it is associated with a primary stimulus is called
secondary or conditioned stimulus. The response to a primary stimulus
is called unconditioned response. The response to a secondary stimulus
is called conditioned response. \cite{Rehman2022Aug}