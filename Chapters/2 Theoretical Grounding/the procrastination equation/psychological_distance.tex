




% CLT Explained
----

Only the present moment is immediately experienced by people.
experiences of the past, present, other locations, people, and alternate realities are psychologically impossible.
Nevertheless, our thoughts, feelings, plans, forecasts, hopes, and counterfactual alternatives fill our minds, shape our decisions, and direct our behavior.
How do we get beyond the present to take into account distant entities?
How do we make long-term plans, consider other people's perspectives, and account for speculative alternatives to reality?

According to the Construal Level Theory (CLT), humans can accomplish this by creating abstract mental construals of distant objects.
Thus, despite the fact that we are unable to experience that which is not present, we are nonetheless able to foresee the future, recall the past, speculate about what may have been, and recall the reactions of others.
Memory, hypothesis, and predictions are all mental constructs that aren't based on actual experience.
They symbolize psychologically remote items and help us see beyond the current circumstances.
A person's subjective perception of something's proximity to or distance from them in the present moment is known as psychological distance.
Thus, psychological distance is egocentric: Its focal point is the present-tense self, and the various ways in which an object could be distanced from that point—in terms of time, location, or social distance—constitute several degrees of psychological distance
there is marked commonality in the way people respond to the different distance dimensions.

Thus, according to CLT,  people overcome different psychological distances by using similar mental construction processes.

Different aspects of psychological distance have been studied in isolation from one another, using different methods and theories to do so.
Tipping points in human evolution include the development of tools that require planning for the future.
Creation of feature-specific tools that require consideration of hypothetical alternatives.
Awareness development that allows distance and perspective perception;

There is marked commonality in the way people respond to the different distance dimensions.
The multidimensional nature of procrastination means that it can be understood in terms of different dimensions or aspects. For example, an event might be seen as being more distant on one dimension (such as time) than another dimension (such as importance). To respond to an event that is increasingly distant on any of those dimensions requires relying more on mental construal and less on direct experience of the event.


As the psychological distance increases, the interpretation becomes more abstract, and as the level of abstraction
increases, so does the psychological distance that people envision. The constructive level broadens or narrows the field of view of the mind. Since these results are communicated constructively, differences in distance should influence predictions, ratings, and actions as well. \cite{Trope2010Apr}
