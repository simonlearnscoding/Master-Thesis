\subsection{Procrastination as a multi-faceted phenomenon}

Despite the advancements in research and development of procrastination measurement,
Haghbin \cite{Haghbin2015} claims that current methods still fail to account for the multidimensional nature of procrastination behaviour and the multifaceted
problems associated with it. Procrastination is not a simple trait or behavior,
but rather is comprised of an array of interrelated elements that interact to create a complex system.
By understanding the nuances of procrastination and its various features, researchers can devise more accurate measures
that better reflect the entirety of procrastination behavior. In tandem with this research,
practitioners also need to work within institutions and organizations in order to create effective
interventions that address underlying causes and target particular aspects of procrastination.

The inability to capture the multifaceted nature of procrastination behaviour when evaluating procrastination
in intervention research and clinical settings calls into serious question the validity of current measurement procedures.
The complexity of procrastination can only be adequately assessed by taking into account a variety of factors,
such as the interrelated elements that contribute to the overall system.

The Steel's Equation is a cognitive model that seeks to address procrastination and the causes behind it.
Developed by psychologist Piers Steel in 2007, this equation proposes four main components that contribute to procrastination:

expectancy, value, impulsiveness, and delay.


\subsection{Steels Procrastination equation}

Steel’s theory of procrastination equation states that an individual’s
tendency to procrastinate is determined by the ratio between the perceived
pleasure and effort of a task. When this ratio is low,
people are far more likely to delay or avoid completing the task. The equation takes into account
both anticipated and impulsiveness in regards to task completion.
For example, an individual who is highly impulsive may actually find pleasure in
a difficult task and thus be more likely to tackle it immediately instead of procrastinating.
Alternatively, someone of low impulsiveness may view a
seemingly easy task as too time-consuming and not worth the effort,
leading to delayed or avoided completion.
Ultimately, the equation can be used to predict the likelihood of an individual procrastinating in any given situation.
By understanding and accounting for factors such as pleasure, effort,
impulsiveness, and anticipated reward,
it is possible to gain greater insight into why certain tasks are more difficult or prone to delay than others.
As such, Steel’s theory provides a useful tool for addressing procrastination and improving productivity.

By understanding how these individual elements interact,
it is possible to develop strategies to increase pleasure or reduce
effort in order to encourage task completion.
With the right adjustments, it may even be possible to completely eliminate,
procrastination from an individual’s life. \cite{Steel2007}

Steel's theory can also be used to understand the potential effects of procrastination on an individual's life.
For instance, procrastinating on important tasks may lead to negative consequences such as missed deadlines or lower grades in school.
Furthermore, it can have a long-term impact on overall productivity and career advancement.
It is therefore essential to address and overcome procrastination in order to achieve success.
By understanding and applying Steel’s theory of procrastination equation, it is possible to gain a better understanding
of why we procrastinate and make the necessary changes to improve our productivity and quality of life.  Through this approach,
it is possible to not just reduce or avoid procrastination but more importantly to embrace and enjoy the tasks ahead.
Not only can Steel’s theory be used as a tool for addressing procrastination, but it also provides valuable insight into
how people make decisions in general. By understanding the factors that lead to task avoidance or completion,
it is possible to gain a better understanding of why certain decisions are made and how those decisions may impact our lives.
As such, Steel’s theory can be seen as a valuable tool for improving decision-making in all aspects of life.
Ultimately, the equation provides an important insight into why we procrastinate,
giving us the tools to make better choices and lead more productive and fulfilling lives.



