
% DEFINITION OF PROCRATIONATION

Procrastination is generally considered to be an irrational
tendency to delay tasks or assignments despite the negative
consequences of such postponement for individuals and organizations\cite{hen2018causes}  \cite{lay1986last} or "a prevalent and prenicious form of self-regulatory failure that is not entirely understood"
Mentions of procrastination have been found in some of the earliest records available, dating back more than 3000 years. \cite{Piers2007}


% PROCRASTINATION IS COMMON

Procrastination is considered a common behavioral pattern,
and there is a growing body of literature discussing this complex phenomenon.\cite{Yan2022}
Research shows that about 80-95 percent of college students procrastinate to some degree, about 75
percent consider themselves  procrastinators, and about 50 pecent are consistently problematic procrastinators. \cite{Steel2007}






%%% CONTINUE HERE %%%%
%https://link.springer.com/chapter/10.1007/978-3-319-20609-7_56
%https://link.springer.com/article/10.1186/s40561-015-0017-8
% TOOLS PRODUCTIVITY NUDGE 

Some of the most common productivity tools for nudge include software such as RescueTime, procrastination.com, and focus@will. These tools can be used to help you create a custom workflow that works for you, and can be very effective in overcoming procrastination.
The state of the art of HCI and productivity is constantly evolving, and there are a variety of tools and resources that can help you stay productive and focused. Some of the best tools for this purpose include software such as RescueTime, procrastination.com, and focus@will. These tools can be used to help you create a custom workflow that works for you, and can be very effective in overcoming procrastination.
Research on nudge and productivity suggests that these tools can be effective in overcoming procrastination.



% THE RELATIONSHIP BETWEEN MOTIVATION AND PROCRASTINATION
There is a lot of research on the relationship between procrastination and motivation,
and it is still an ongoing area of exploration. Some of the most recent research on this topic has been conducted by Piers Steel, who has developed a model known as the "motivation equation". This model suggests that there are four factors that influence motivation:


- Ability: The ability to complete a task

- Motivation: The desire to complete a task

- Opportunity: The opportunity to complete a task

- Expectancy: The belief that completing a task will lead to a desired outcome


According to this model, motivation is influenced by both ability and expectancy. If you believe that you are not able to complete a task, or if you do not believe that completing a task will lead to the desired outcome, your motivation will be reduced.



There are a few key differences between nudge and gamification. Nudge theory is based on the idea that people can be subtly influenced into making certain choices, while gamification relies on more explicit rewards and punishments to motivate people. Nudge theory is also more focused on changing behaviour, while gamification is more focused on increasing engagement or motivation.


- The difference between nudge and gamification

- How to use nudge theory to increase productivity

- How to use gamification to increase engagement


The paper discusses the difference between nudge and gamification, and how each can be used to increase productivity or engagement. It cites research from John Perry, Piers Steel, and Tim Pychyl on procrastination, and explains how each researcher has developed different strategies for overcoming it. Finally, it summarizes the motivation equation developed by Piers Steel, which suggests that there are four factors that influence motivation: ability, motivation, opportunity, and expectancy.



How do procrastination relate to motivation?

Both procrastination and motivation are related to goal setting. Procrastination occurs when people do not set goals for themselves, or when they set unrealistic goals that are difficult to achieve. Motivation occurs when people set realistic goals and have the ability to achieve them.


According to Tom Perry, the relationship between motivation and procrastination is complex. Motivation can lead to procrastination if people set unrealistic goals for themselves, or if they do not believe that they are able to complete a task. However, procrastination can also lead to motivation if people are able to overcome their fear of failure and start taking small steps towards their goal.


Procrastination is generally seen as a common behavioral tendency, and there are a growing number of literatures discussing this complex phenomenon.


Write an essay in which you discuss the statement "procrastination is the main productivity killer."


Your essay should include the following:

- An overview of the research on procrastination and motivation

- The impact of procrastination on productivity

- The impact of motivation on productivity

- How to overcome procrastination and increase productivity

- What can be done to increase motivation


Procrastination has long been considered a productivity killer, and recent research has confirmed the link between procrastination and lower levels of productivity. Procrastination is defined as wasting time on activities that are not related to the task at hand, or delaying tasks without good reason. It can be caused by a lack of motivation, fear of failure, lack of goal setting, or even just a desire to avoid the task.


Recent research has demonstrated that procrastination can have a significant impact on productivity. Studies have shown that people who procrastinate are more likely to miss deadlines, be less productive at their jobs, and experience higher levels of stress. This can lead to


- Research from John Perry suggests that people who procrastinate are more likely to miss deadlines and be less productive at their jobs.


- A study by Piers Steel showed that people who procrastinate are more likely to experience higher levels of stress.

- Tim Pychyl's research indicates that procrastination can lead to a cycle of anxiety and low productivity.




